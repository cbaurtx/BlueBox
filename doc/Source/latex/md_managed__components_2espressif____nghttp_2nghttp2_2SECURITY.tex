\chapter{Security Process }
\hypertarget{md_managed__components_2espressif____nghttp_2nghttp2_2SECURITY}{}\label{md_managed__components_2espressif____nghttp_2nghttp2_2SECURITY}\index{Security Process@{Security Process}}
\label{md_managed__components_2espressif____nghttp_2nghttp2_2SECURITY_autotoc_md208}%
\Hypertarget{md_managed__components_2espressif____nghttp_2nghttp2_2SECURITY_autotoc_md208}%


If you find a vulnerability in our software, please report it via Git\+Hub "{}\+Private vulnerability reporting"{} feature at \href{https://github.com/nghttp2/nghttp2/security}{\texttt{https\+://github.\+com/nghttp2/nghttp2/security}} instead of submitting issues on github issue page. It is a standard practice not to disclose vulnerability information publicly until a fixed version is released, or mitigation is worked out.

If we identify that the reported issue is really a vulnerability, we open a new security advisory draft using \href{https://github.com/nghttp2/nghttp2/security}{\texttt{Git\+Hub security feature}} and discuss the mitigation and bug fixes there. The fixes are committed to the private repository.

We write the security advisory and get CVE number from Git\+Hub privately. We also discuss the disclosure date to the public.

We make a new release with the fix at the same time when the vulnerability is disclosed to public.

At least 7 days before the public disclosure date, we open a new issue on \href{https://github.com/nghttp2/nghttp2/issues}{\texttt{nghttp2 issue tracker}} which notifies that the upcoming release will have a security fix. The {\ttfamily SECURITY} label is attached to this kind of issue. The issue is not opened if a vulnerability is already disclosed, and it is publicly known that nghttp2 is affected by that.

Before few hours of new release, we merge the fixes to the master branch (and/or a release branch if necessary) and make a new release. Security advisory is disclosed on Git\+Hub. 